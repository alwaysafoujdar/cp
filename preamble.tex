% preamble.tex
\usepackage[utf8]{inputenc}
\usepackage[T1]{fontenc}
\usepackage{lmodern}
\usepackage{microtype}
\usepackage{geometry}
\geometry{a4paper, margin=1in}

% Maths
\usepackage{amsmath,amssymb,amsthm}
\usepackage{mathtools}
\usepackage{mathrsfs}

% Graphics and links
\usepackage{graphicx}
\usepackage{hyperref}
\hypersetup{colorlinks=true, linkcolor=blue, urlcolor=blue}

% Nice boxes for 'chill vibes' explanations & problems
\usepackage{tcolorbox}
\tcbset{colback=white, colframe=black!50, boxrule=0.6pt, arc=3pt}

\newtcolorbox{friendly}[1][]{
  title=\textbf{Friendly note}, fonttitle=\bfseries, #1
}
\newtcolorbox{examplebox}[1][]{
  title=\textbf{Example}, fonttitle=\bfseries, #1
}
\newtcolorbox{problembox}[1][]{
  title=\textbf{Practice Problem}, fonttitle=\bfseries, #1
}

\usepackage{imakeidx}
\makeindex[options=-s mystyle.ist] % more control

% Code listing (for snippets)
\usepackage{listings}
\lstset{basicstyle=\ttfamily\small,breaklines=true}

% Section styling
\usepackage{titlesec}
\titleformat{\chapter}[hang]{\Huge\bfseries}{\thechapter.}{1em}{}
\titleformat{\section}[hang]{\large\bfseries}{\thesection.}{0.5em}{}

% Custom commands
\newcommand{\idea}[1]{\begin{friendly}#1\end{friendly}}
\newcommand{\example}[1]{\begin{examplebox}#1\end{examplebox}}
\newcommand{\problem}[1]{\begin{problembox}#1\end{problembox}}

% Bibliography settings (if using bibtex)
% \usepackage{natbib}  % optional

% End preamble
